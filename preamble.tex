\providecommand{\event}{ACT 2025} % Name of the event you are submitting to

\usepackage{iftex}

\ifpdf
  \usepackage{underscore}         % Only needed if you use pdflatex.
  \usepackage[T1]{fontenc}        % Recommended with pdflatex
\else
  \usepackage{breakurl}           % Not needed if you use pdflatex only.
\fi

\usepackage[utf8]{inputenc}
\usepackage[english]{babel}
\usepackage[font=small,labelfont=bf]{caption}

\usepackage{caption}
\usepackage{paralist} % for compactitem
\usepackage{csquotes}

\usepackage{amsmath}
\usepackage{amssymb}
\usepackage{amsthm}

\usepackage{mathtools}
\usepackage{IEEEtrantools}

\usepackage{eucal}
\usepackage{bbm}
\usepackage{stmaryrd} %fatsemi
\usepackage{bbm} % blackboard bold numbers

\usepackage{trimclip}

\interdisplaylinepenalty = 2500

\usepackage{scalerel}

\usepackage[dvipsnames]{xcolor}
\definecolor{darkgreen}{RGB}{35, 89, 52}
\definecolor{paleorange}{RGB}{255, 236, 207}
\definecolor{paleyellow}{RGB}{255, 252, 217}
\definecolor{paleteal}{RGB}{223, 245, 243}
\definecolor{palered}{RGB}{255, 234, 232}
\definecolor{palepurple}{RGB}{238, 219, 255}

\usepackage{graphicx}
\graphicspath{ {./images/} }

\usepackage{hyperref}
\hypersetup{
	colorlinks=true,
	linkcolor=darkgreen,
	filecolor=magenta,      
	urlcolor=MidnightBlue,
	citecolor=darkgreen,
	pdftitle={} %Add title
}

\usepackage{quiver} % For commutative diagrams

\usepackage{enumitem} % For list spacing

\usepackage{cleveref} % load cleverref in the end, after all other packages


% \usepackage[
% backend=biber,
% maxbibnames=50,
% style=eptcs]{biblatex}
% \addbibresource{generic.bib}

%NOTE: does not work for Bibtex. EPTCS citations style is only available in bibtex format.
%Remove In: from journal articles
% \renewbibmacro{in:}{%
%   \ifentrytype{article}{}{\printtext{\bibstring{in}\intitlepunct}}}


%%% Theorem Styles %%%
\theoremstyle{definition}

\newtheorem{counter}{Definition}[section]

\newtheorem{definitionx}[counter]{Definition}
\newenvironment{definition}
  {\pushQED{\qed}\renewcommand{\qedsymbol}{$\triangleleft$}\definitionx}
  {\popQED\enddefinitionx}

\theoremstyle{plain} 
\newtheorem{lemma}[counter]{Lemma}

\theoremstyle{plain} 
\newtheorem{proposition}[counter]{Proposition}

\theoremstyle{plain}
\newtheorem{theorem}[counter]{Theorem}

\theoremstyle{plain}
\newtheorem{question}[counter]{Question}

\theoremstyle{plain}
\newtheorem{problem}[counter]{Problem}

\theoremstyle{remark}
\newtheorem{examplex}[counter]{Example}
\newenvironment{example}
  {\pushQED{\qed}\renewcommand{\qedsymbol}{$\diamond$}\examplex}
  {\popQED\endexamplex}

\newtheorem*{excont}{Example \continuation}
\newcommand{\continuation}{??}
\newenvironment{continueexample}[1]
{\renewcommand{\continuation}{\ref{#1}}\excont[continued]}
{\endexcont}

\theoremstyle{remark}
\newtheorem{counterexamplex}[counter]{Counterexample}
\newenvironment{counterexample}
  {\pushQED{\qed}\renewcommand{\qedsymbol}{$\diamond$}\counterexamplex}
  {\popQED\endcounterexamplex}

\theoremstyle{remark}
\newtheorem{remark}[counter]{Remark}

%%% Convenience Macros %%
\newcommand{\mc}[1]{\mathcal{#1}}
\newcommand{\msf}[1]{\mathsf{#1}}
\newcommand{\mbf}[1]{\mathbf{#1}}
\newcommand{\mbb}[1]{\mathbb{#1}} 

% Sets
\newcommand{\Reals}{\mathbb{R}}

\newcommand{\sub}{\subseteq}
\newcommand{\then}{\fatsemi}

\newcommand{\upper}{\mathsf{U}}

\newcommand{\Ob}{\mathsf{Ob}}

\newcommand{\argmax}[1]{\underset{#1}{\text{argmax }}}

% Categories
\newcommand{\functorArr}{\to}

\newcommand{\cat}[1]{\mathcal{#1}}
\newcommand{\functor}[1]{\mathsf{#1}}
\newcommand{\twocat}[2]{\mathbb{#1}\mathsf{#2}}

\newcommand{\C}{\cat{C}}
\newcommand{\D}{\cat{D}}
\newcommand{\E}{\cat{E}}
\newcommand{\F}{\cat{F}}
\newcommand{\R}{\cat{R}}
\newcommand{\Q}{\cat{Q}}
\newcommand{\V}{\cat{V}}
\newcommand{\W}{\cat{W}}

\newcommand{\strict}{_{\text{s}}}
\newcommand{\Cs}{\C\strict}
\newcommand{\Vs}{\V\strict}

\newcommand{\Kl}{\mathsf{Kl}} % Kleisli category
\newcommand{\Para}{\mathbf{Para}}
\newcommand{\ExPara}{\mathbf{ExPara}}
\newcommand{\ParaOf}[1]{\Para(#1)}
\newcommand{\ExPaOf}[2]{\ExPara(#1; #2)}

% Regular categories
\newcommand{\Bool}{\mathsf{Bool}}
\newcommand{\Pos}{\mathsf{Pos}}
\newcommand{\Set}{\mathsf{Set}}
\newcommand{\Top}{\mathsf{Top}}
\newcommand{\smcat}{\mathsf{cat}} %The category of small categories and functors
\newcommand{\Meas}{\mathsf{Meas}}
\newcommand{\DP}{\mathsf{DP}}
\newcommand{\Stoch}{\mathsf{Stoch}}
\newcommand{\BorelStoch}{\mathsf{BorelStoch}}

% Markov Categories
\newcommand{\Markov}{\mathsh{Markov}}
\newcommand{\detSubCat}{_\text{det}}
\newcommand{\copyMor}{\mathsf{cp}}
\newcommand{\delMor}{\mathsf{del}}


\newcommand{\SigAlg}{\Sigma}
\newcommand{\interval}[2]{[#1, #2]}

% monoidal product in certain categories
\newcommand{\prodSmcat}{\times}
\newcommand{\prodW}{\otimes}

% general functors
\newcommand{\functorF}{\functor{F}}

% Named functors
\newcommand{\ProbDP}{\functor{ProbDP}}
\newcommand{\PowNoEmpty}{\functor{Pow \backslash \varnothing}}
\newcommand{\Prob}{\functor{P}}
\newcommand{\Arr}{\functor{Arr}}
\newcommand{\TwiArr}{\functor{TwiArr}}
\newcommand{\leftFunctor}{\functor{L}}

% 2-cateogries
\newcommand{\CCat}{\twocat{C}{at}} %The 2-category of small categories, functors and natural transformations.
\newcommand{\MMoncat}{\twocat{M}{oncat}} %The 2-category of small monoidal categories, lax monoidal functors, and monoidal transformations
\newcommand{\TTwocat}{\mathbbm{2}\mathsf{cat}} %The 2-category of 2-categories, 2-functors, and 2-natural transformations. (i.e Cat-cat).
\newcommand{\VCat}{\twocat{V}{cat}}
\newcommand{\WCat}{\twocat{W}{cat}}

\newcommand{\CC}{\twocat{C}{}}
\newcommand{\DD}{\twocat{D}{}}
\newcommand{\EE}{\twocat{E}{}}

% Diagramitic composition symbols
% new \oset macro for less vertical space:
\makeatletter
\newcommand{\oset}[3][0ex]{%
  \mathrel{\mathop{#3}\limits^{
    \vbox to#1{\kern-2.2\ex@
    \hbox{$\scriptstyle#2$}\vss}}}}
\makeatother
\newcommand{\vertthen}{{\oset{\diamond}{\text{{\clipbox{0 0 0 3.4pt}{$\fatsemi$}}}}}}
\newcommand{\horthen}{{\oset{\star}{\text{{\clipbox{0 0 0 3.4pt}{$\fatsemi$}}}}}}

\newcommand{\op}[1]{#1^{\text{op}}}
\newcommand{\id}{\mathsf{id}}
\newcommand{\Id}{\mathsf{Id}}

% for monads
\newcommand{\monadM}{\functor{M}}
\newcommand{\monadUnit}{\eta}
\newcommand{\monadMul}{\mu}
\newcommand{\monadTup}{(\monadM, \monadMul, \monadUnit)}
\newcommand{\symMonadMor}{\nabla}
\newcommand{\symMonadMorOf}[2]{\symMonadMor_{#1,#2}}

% slice functors and utilities
\newcommand{\sliceWith}[1]{/#1}
\newcommand{\sliceFunctor}{\sliceWith{(-)}}

\newcommand{\WsliceWith}[1]{\W\sliceWith{#1}}
\newcommand{\sliceFunctorW}{\W \sliceFunctor}
\newcommand{\WsliceCat}{{\W\sliceWith{(-)}}}

\newcommand{\sliceFunWunit}{\sliceFunctorW_\epsilon}
\newcommand{\sliceFunWnat}{\widetilde \sliceFunctorW}
\newcommand{\sliceFunWnatOf}[2]{\sliceFunWnat_{#1, #2}}

\newcommand{\preCompose}[1]{{#1 \then (-)}}
\newcommand{\postCompose}[1]{(-) \then #1}

\newcommand{\precomposeFunctor}[1]{\preCompose{#1}}
\newcommand{\postcomposeFunctor}[1]{\postCompose{#1}}

\newcommand{\unitObj}{1}
\newcommand{\unitOf}[1]{\unitObj_#1}
\newcommand{\unitW}{I_\W}
\newcommand{\idUnitW}{\id_{\unitW}}
\newcommand{\unitCat}{\unitOf{{\CCat}}}

% posets
\newcommand{\posetP}{P}
\newcommand{\posetQ}{Q}
\newcommand{\posetleq}{\preceq}

% arrow styles
\newcommand{\elementTo}{\mapsto}
